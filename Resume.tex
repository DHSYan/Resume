\documentclass{article}
\usepackage[cm]{fullpage}
\usepackage{color}
\usepackage{hyperref}
\usepackage{fontawesome}
\usepackage{CJK}

\hypersetup{breaklinks=true,%
colorlinks=true,%
linkcolor=cyan,%
urlcolor=MyDarkBlue}

\definecolor{MyDarkBlue}{rgb}{0,0.0,0.45}

%%%%%%%%%%%%%%%%%%%%%%%%%%
% Formatting parameters  %
%%%%%%%%%%%%%%%%%%%%%%%%%%

\newlength{\tabin}
\setlength{\tabin}{1em}
\newlength{\secsep}
\setlength{\secsep}{0.13cm}

\setlength{\parindent}{0in}
\setlength{\parskip}{0in}
\setlength{\itemsep}{0in}
\setlength{\topsep}{0in}
\setlength{\tabcolsep}{0in}

\definecolor{contactgray}{gray}{0.2}
\pagestyle{empty}

%%%%%%%%%%%%%%%%%%%%%%%%%%
%  Template Definitions  %
%%%%%%%%%%%%%%%%%%%%%%%%%%

\newcommand{\lineunder}{\vspace*{-8pt} \\ \hspace*{-6pt} \hrulefill \\ \vspace*{-15pt}}
\newcommand{\name}[1]{\begin{center}\textsc{\Huge#1}\\\end{center}}
\newcommand{\program}[1]{\begin{center}\textsc{#1}\end{center}}
\newcommand{\contact}[1]{\begin{center}\color{contactgray}{\small#1}\end{center}}

\newenvironment{tabbedsection}[1]{
  \begin{list}{}{
      \setlength{\itemsep}{0pt}
      \setlength{\labelsep}{0pt}
      \setlength{\labelwidth}{0pt}
      \setlength{\leftmargin}{\tabin}
      \setlength{\rightmargin}{\tabin}
      \setlength{\listparindent}{0pt}
      \setlength{\parsep}{0pt}
      \setlength{\parskip}{0pt}
      \setlength{\partopsep}{0pt}
      \setlength{\topsep}{#1}
    }
  \item[]
}{\end{list}}

\newenvironment{nospacetabbing}{
    \begin{tabbing}
}{\end{tabbing}\vspace{-1.2em}}

\newenvironment{resume_header}{}{\vspace{0pt}}


\newenvironment{resume_section}[1]{
  \filbreak
  \vspace{2\secsep}
  \textsc{\large#1}
  \lineunder
  \begin{tabbedsection}{\secsep}
}{\end{tabbedsection}}

\newenvironment{resume_subsection}[2][]{
  \textbf{#2} \hfill {\footnotesize #1} \hspace{-5.1em}
  \begin{tabbedsection}{0.5\secsep}
}{\end{tabbedsection}}

\newenvironment{subitems}{
  \renewcommand{\labelitemi}{-}
  \begin{itemize}
      \setlength{\labelsep}{1em}
}{\end{itemize}}

\newenvironment{resume_employer}[4]{
  \vspace{\secsep}
  \textbf{#1} {\footnotesize #2} \hfill {\footnotesize #3} \hspace{1em} {\footnotesize#4} \hspace{-5em}
  \vspace{\secsep}
  \begin{tabbedsection}{0pt}
  \begin{subitems}
}{\end{subitems}\end{tabbedsection}}


%%%%%%%%%%%%%%%%%%%%%%%%%%
%     Start Document     %
%%%%%%%%%%%%%%%%%%%%%%%%%%

\begin{document}
\begin{CJK*}{UTF8}{bsmi}
\fontsize{12pt}{12pt}\selectfont

\begin{resume_header}
  \name{顏鼎翰(Sean Yan)}
% \program{1B Honours Mathematics}
  \contact{
    \faLinkedinSquare \hspace{0.01cm} \href{https://www.linkedin.com/in/dhsyan/}{linkedin.com/in/dhsyan}
    \hspace{0.5cm} 
    \faEnvelopeSquare \hspace{0.01cm} \href{mailto:dhsyan@student.ubc.ca}{dhsyan@student.ubc.ca}
    \hspace{0.5cm} 
    \faPhoneSquare \hspace{0.01cm} \href{tel:0905518422}{0905-518-422}
    \hspace{0.5cm}
    % \faGlobe \hspace{0.01cm} \href{https://github.com/DHSYan}{Personal Website}
    \hspace{0.5cm}
    \faGithubSquare \hspace{0.01cm} \href{https://github.com/DHSYan}{DHSYan}
  }
\end{resume_header}

\begin{resume_section}{技能專長}
  \begin{nospacetabbing}

  \textbf{程式語言:}  \= Javascript/Typescript, C/C++, Java, Python, Rust, Scheme, Assembly, HTML/CSS, LaTeX \\*
  \textbf{Tools:} \> React, NextJS, Flask, ExpressJS, Selenium, Pandas, Nix/NixOS, Git/GitHub, Linux, Vim, Bash, Testing\\* 
                  \> VSCode, SSH, VirtualBox/VMware, Automation, Raspberry Pi, Excel, Word, MacOS, Windows

  \end{nospacetabbing}

\end{resume_section}

\begin{resume_section}{工作經驗}
    \begin{resume_employer}{數學與英文家教}{明光義塾}{台中市, 台灣}{05/2023 - 07/2023}
        \item 教授1至12年級學生數學與英文。
        \item 協助學生將數學校內考試分數從46\%提升至96\%。
        \item 透過生動有趣的教學方式,幫助學生在數學領域建立自信。
        \item 撰寫高品質的報告,協助家長更了解學生在課程中的表現,並提供持續自我提升的建議。
    \end{resume_employer}
\end{resume_section}

\begin{resume_section}{專案}
    \begin{resume_subsection}[]{\href{}{PRIMPL Assembler, SIMPL Compiler}}
        \begin{subitems}
        \item 為指令式語言和組合語言寫編譯器和組譯器,其中設計了堆疊及語法剖析器。
        \item 透過此作品,深入了解編程和資工的核心概念,例如內存管理、低級電腦結構、數據結構和算法。
        \end{subitems}
    \end{resume_subsection}

  % \begin{resume_subsection}[]{\href{}{Self Driving Arduino Car}}
  %   \begin{subitems}
  %     \item Engineered a self driving car that follows line with Arduino. 
  %     \item Acquired skills of a team player, and team leader in the Software Department
  %   \end{subitems}
  % \end{resume_subsection}

  \begin{resume_subsection}[]{\href{https://github.com/TwinkletoesZen/nix-config}{NixOS 配置;基礎架構即程式碼(重灌無懼!)\faGithub}}
      \begin{subitems}
      \item 開發聲明式且可重現的 NixOS 配置,確保不同系統間作業系統設定一致。
      \item 實作基礎設施即程式碼,自動化系統重建流程,省去手動設定。
      \item 透過日常使用 NixOS,深入理解 Linux 系統,效率提升100\%。
      \end{subitems}
  \end{resume_subsection}

  \begin{resume_subsection}[]{\href{https://github.com/TwinkletoesZen/.dotfiles}{.dotfiles - Personalized Software Developement Workflow \faGithub}}
    \begin{subitems}
      \item 積極主動的熱情項目始於對掌握軟件開發過程的熱愛和目標。
      \item 具有使用NeoVim 和Lua、Bash 腳本、軟體自動化和Tmux的能力。
      \item 通過100 多次commit,培養了管理Git/Github 工作流程的強大技能。
    \end{subitems}
  \end{resume_subsection}

  \begin{resume_subsection}[]{\href{https://github.com/TwinkletoesZen/.dotfiles}{終端聊天應用程式(100\%使用C語言開發)}}
      \begin{subitems}
      \item 使用 C 語言、TCP、UDP、Socket 與多執行緒技術開發終端聊天應用程式。
      \item 支援多用戶登入、私人訊息、廣播訊息及媒體傳輸功能。
      \item 習得多執行緒程式設計、網路程式設計及低階程式設計技能。
      \end{subitems}
  \end{resume_subsection}

  % \begin{resume_subsection}[]{\href{https://github.com/TwinkletoesZen/Auto-HealthCheck-Completer}{Auto Form Filler (Web Scraper) \faGithub}}
  %   \begin{subitems}
  %     \item 使用Python和Selenium設計網路爬蟲通過終端與網絡交互,在疫情期間自動提交日常健康表格,優化了學生重複的日常表格填寫過程。
  %   \end{subitems}
  % \end{resume_subsection}
\end{resume_section}


 \begin{resume_section}{志工經歷}
  \begin{resume_employer}{農場志工}{台灣在地農場}{苗栗縣, 台灣}{07/2022 - 至今}
    \item 在台灣的在地農場工作,協助種植樹木、香蕉樹及維護花園。
  \end{resume_employer}
  \begin{resume_employer}{學生議會議長}{博德威中學}{加拿大溫哥華北區}{02/2022 - 06/2022}
    \item 主持並管理所有學生議會的會議,並培養了溝通與人際互動能力。
  \end{resume_employer}
  \begin{resume_employer}{英文教師}{Team Alpha}{台中市, 台灣}{07/2020}
    \item 幾乎在無監督下,自願每天教授台灣小學生英文8小時。
  \end{resume_employer}
\end{resume_section}
 
% \begin{resume_section}{Awards}
%   \begin{nospacetabbing}
%     \textbf{社區服務獎學金:} \= 因對學校社區做出傑出貢獻而獲獎。\\
%     \textbf{畢業生代表:} \> 由Bodwell 高中授予2022 屆畢業生。\\
%   \end{nospacetabbing}
% \end{resume_section}

\begin{resume_section}{學歷}
    \begin{resume_subsection}[09/2023 - 至今]{英屬哥倫比亞大學}
        \begin{nospacetabbing}
            \textbf{學位:} \= 理學士 \\*
            \textbf{主修:} \> 綜合科學主修 \\*
            \textbf{GPA:} \> 4.33/4.33 \\*
      % \textbf{Courses:} \>Advanced Data Structures \& Algorithm, Advanced Designing Functional Programs.
      %                   \\*\> Tools and Techniques for Software Development, Linear Algebra 1.\\*
      % \textbf{Extra-curricular:} \= Ultimate Frisbee
    \end{nospacetabbing}
  \end{resume_subsection}
\end{resume_section}

\end{CJK*}  
\end{document}
