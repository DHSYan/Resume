\documentclass{article}
\usepackage[cm]{fullpage}
\usepackage{color}
\usepackage{hyperref}
\usepackage{fontawesome}
\usepackage{CJK}

\hypersetup{breaklinks=true,%
colorlinks=true,%
linkcolor=cyan,%
urlcolor=MyDarkBlue}

\definecolor{MyDarkBlue}{rgb}{0,0.0,0.45}

%%%%%%%%%%%%%%%%%%%%%%%%%%
% Formatting parameters  %
%%%%%%%%%%%%%%%%%%%%%%%%%%

\newlength{\tabin}
\setlength{\tabin}{1em}
\newlength{\secsep}
\setlength{\secsep}{0.13cm}

\setlength{\parindent}{0in}
\setlength{\parskip}{0in}
\setlength{\itemsep}{0in}
\setlength{\topsep}{0in}
\setlength{\tabcolsep}{0in}

\definecolor{contactgray}{gray}{0.2}
\pagestyle{empty}

%%%%%%%%%%%%%%%%%%%%%%%%%%
%  Template Definitions  %
%%%%%%%%%%%%%%%%%%%%%%%%%%

\newcommand{\lineunder}{\vspace*{-8pt} \\ \hspace*{-6pt} \hrulefill \\ \vspace*{-15pt}}
\newcommand{\name}[1]{\begin{center}\textsc{\Huge#1}\\\end{center}}
\newcommand{\program}[1]{\begin{center}\textsc{#1}\end{center}}
\newcommand{\contact}[1]{\begin{center}\color{contactgray}{\small#1}\end{center}}

\newenvironment{tabbedsection}[1]{
  \begin{list}{}{
      \setlength{\itemsep}{0pt}
      \setlength{\labelsep}{0pt}
      \setlength{\labelwidth}{0pt}
      \setlength{\leftmargin}{\tabin}
      \setlength{\rightmargin}{\tabin}
      \setlength{\listparindent}{0pt}
      \setlength{\parsep}{0pt}
      \setlength{\parskip}{0pt}
      \setlength{\partopsep}{0pt}
      \setlength{\topsep}{#1}
    }
  \item[]
}{\end{list}}

\newenvironment{nospacetabbing}{
    \begin{tabbing}
}{\end{tabbing}\vspace{-1.2em}}

\newenvironment{resume_header}{}{\vspace{0pt}}


\newenvironment{resume_section}[1]{
  \filbreak
  \vspace{2\secsep}
  \textsc{\large#1}
  \lineunder
  \begin{tabbedsection}{\secsep}
}{\end{tabbedsection}}

\newenvironment{resume_subsection}[2][]{
  \textbf{#2} \hfill {\footnotesize #1} \hspace{-5.1em}
  \begin{tabbedsection}{0.5\secsep}
}{\end{tabbedsection}}

\newenvironment{subitems}{
  \renewcommand{\labelitemi}{-}
  \begin{itemize}
      \setlength{\labelsep}{1em}
}{\end{itemize}}

\newenvironment{resume_employer}[4]{
  \vspace{\secsep}
  \textbf{#1} {\footnotesize #3} \hfill {\footnotesize#4} \hspace{-1em} \\
  \small{#2}  
  \vspace{\secsep}
  \begin{tabbedsection}{0pt}
  \begin{subitems}
}{\end{subitems}\end{tabbedsection}}


%%%%%%%%%%%%%%%%%%%%%%%%%%
%     Start Document     %
%%%%%%%%%%%%%%%%%%%%%%%%%%

\begin{document}
\begin{CJK*}{UTF8}{bsmi}

\begin{resume_header}
  \name{顏鼎翰(Sean Yan)}
% \program{1B Honours Mathematics}
  \contact{
    \faLinkedinSquare \hspace{0.01cm} \href{https://www.linkedin.com/in/dhsyan/}{linkedin.com/in/dhsyan}
    \hspace{0.5cm} 
    \faEnvelopeSquare \hspace{0.01cm} \href{mailto:sean.yan@uwaterloo.ca}{sean.yan@uwaterloo.ca}
    \hspace{0.5cm} 
    \faPhoneSquare \hspace{0.01cm} \href{tel:7782519186}{0905518422}
    \hspace{0.5cm}
    \faGlobe \hspace{0.01cm} \href{https://twinkletoes5.netlify.app}{個人網站}
    \hspace{0.5cm}
    \faGithubSquare \hspace{0.01cm} \href{https://github.com/DHSYan}{DHSYan}
  }
\end{resume_header}

\begin{resume_section}{Skills}
  \begin{nospacetabbing}

  \textbf{Languages:}  \= C, Python, Rust, Scheme(Racket), MMIX(Assembly Like), HTML/CSS, LaTeX \\*
  \textbf{Tools:} \> React, NextJS, Flask, ExpressJS, Selenium, Tkinter, Git/GitHub, Linux, Vim, Bash, Testing\\* 
                  \> VSCode, SSH, VirtualBox/VMware, Automation, Raspberry Pi, Excel, Word, MacOS, Windows

  \end{nospacetabbing}


\end{resume_section}

\begin{resume_section}{Projects}
  \begin{resume_subsection}[]{\href{}{PRIMPL Assembler, SIMPL Compiler}}
   \begin{subitems}
     \item 為指令式語言和組合語言寫編譯器和組譯器,其中設計了堆疊及語法剖析器。
     \item 透過此作品,深入了解編程和資工的核心概念,例如內存管理、低級電腦結構、數據結構和算法。
    \end{subitems}
  \end{resume_subsection}

  \begin{resume_subsection}[]{\href{https://github.com/TwinkletoesZen/.dotfiles}{.dotfiles - Software Developement Workflow \faGithub}}
    \begin{subitems}
      \item 積極主動的熱情項目始於對掌握軟件開發過程的熱愛和目標。
      \item 具有使用NeoVim 和Lua、Bash 腳本、軟體自動化和Tmux的能力。
      \item 通過100 多次commit,培養了管理Git/Github 工作流程的強大技能。
    \end{subitems}
  \end{resume_subsection}

  \begin{resume_subsection}[]{\href{https://github.com/TwinkletoesZen/Personal-Website}{Personal Website \faGithub}}
    \begin{subitems}
      \item 使用HTML 和CSS 開發了個人網站來展示作品及簡介。
      \item 掌握CSS概念、Flexbox等前端設計基礎知識。
      \item 培養了對網絡原則的核心理解,例如網站架構、DOM、瀏覽器和網絡伺服器。
    \end{subitems}
  \end{resume_subsection}

  \begin{resume_subsection}[]{\href{https://github.com/TwinkletoesZen/Auto-HealthCheck-Completer}{Auto Form Filler (Web Scraper) \faGithub}}
    \begin{subitems}
      \item 使用Python和Selenium設計網路爬蟲通過終端與網絡交互,在疫情期間自動提交日常健康表格,優化了學生重複的日常表格填寫過程。
    \end{subitems}
  \end{resume_subsection}
\end{resume_section}


\begin{resume_section}{Experiences}
  \begin{resume_employer}{高中學生會議長}{Bodwell High School}{North Vancouver, BC}{02/2022 - 06/2022}
    \item 領導和管理學生會召開的所有會議,並培養溝通和人際交往能力。
    \item 指導和組織校內活動,包括全校體育賽事和才藝表演,並形成協作能力。
    \item 通過富有同情心和有效的演講者和溝通者來指導議會的溝通。
  \end{resume_employer}

  \begin{resume_employer}{高中學生會通訊部部長}{Bodwell High School}{North Vancouver, BC}{02/2022}
    \item 領導通訊部團隊,在科技、藝術和營銷方面指導學生會。
    \item 在該部工作一年後被選為通訊部部長。
  \end{resume_employer}

  \begin{resume_employer}{英文老師}{Team Alpha}{Taichung, TW}{07/2020}
    \item 在最少的監督下,自願每天教當地台灣小學生英語 8 小時。
    \item 獨立管理截止日期,並組織課程編程、內容、規劃和課堂設置。
    \item 在與同齡人合作以提供最佳學習體驗的同時,培養了良好的人際交往能力。
  \end{resume_employer}

\end{resume_section}
  
\begin{resume_section}{Awards}
  \begin{nospacetabbing}
    \textbf{社區服務獎學金:} \= 因對學校社區做出傑出貢獻而獲獎。\\
    \textbf{畢業生代表:} \> 由Bodwell 高中授予2022 屆畢業生。\\
  \end{nospacetabbing}
\end{resume_section}

\begin{resume_section}{Education}
  \begin{resume_subsection}[07/2022 - Present]{University of Waterloo}
    \begin{nospacetabbing}
      \textbf{科系:} \= 資優數學系\\*
      \textbf{課程:} \> 資優數據結構及算法  \& 資優設計功能程序  \& 軟件開發工具和技術  \& 線性代數 \\*
      \textbf{獎項:} \> 校長獎學金\\*
    \end{nospacetabbing}
  \end{resume_subsection}
\end{resume_section}

\end{CJK*}  
\end{document}

